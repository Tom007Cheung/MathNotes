\documentclass[12pt, twoside, a4paper]{book}
% \doumentclass[openany] 可以去掉空白页...............
\usepackage{CJK}
\usepackage{graphicx}
% \usepackage{unnumberedtotoc}
\setcounter{tocdepth}{3}

\begin{document}
%%%%%%%%%%%%%%%%%%%%%%%%%%%%%%%%%%%%%%%%%%%%%%%%%%%%
\begin{CJK}{UTF8}{gbsn}
%%%%%%%%%%%%%%%%%%%%%%%%%%%%%%%%%%%%%%%%%%%%%%%%%%%%
\frontmatter
\newcounter{y}
\title{考研数学笔记}
\author{张亚栋}
\maketitle

\part*{前言}
\addcontentsline{toc}{part}{前言}
	\centering
	\paragraph{A Renaissance person, also a Rookie.}
%%%%%%%%%%%%%%%%%%%%%%%%%%%%%%%%%%%%%%%%%%%%%%%%%%%%  
%%%%%%%%%%%%%%%%%%%%%%%%%%%%%%%%%%%%%%%%%%%%%%%%%%%%
\mainmatter
\part*{高等数学·第七版}
\addcontentsline{toc}{part}{高等数学·第七版}
	\part*{上册}
    	\chapter*{第一章 函数与极限}
        	\section*{第一节 映射与函数}
            	\subsection*{一、映射}
                	\subsubsection*{1. 概念}
                    	\paragraph*{定义:}
    \part*{下册}
\part*{线性代数·第六版}
\addcontentsline{toc}{part}{线性代数·第六版}

\backmatter
\end{CJK}

\end{document}